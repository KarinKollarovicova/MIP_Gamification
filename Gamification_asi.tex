% Metódy inžinierskej práce

\documentclass[10pt,twoside,english,a4paper]{article}

\usepackage[english]{babel}
%\usepackage[T1]{fontenc}
\usepackage[IL2]{fontenc} % lepšia sadzba písmena Ľ než v T1
\usepackage[utf8]{inputenc}
\usepackage{graphicx}
\usepackage{url} % príkaz \url na formátovanie URL
\usepackage{hyperref} % odkazy v texte budú aktívne (pri niektorých triedach dokumentov spôsobuje posun textu)

\usepackage{cite}
%\usepackage{times}

\pagestyle{headings}

\title{Gamification-How to motivate students in an online environment\thanks{Semestrálny projekt v predmete Metódy inžinierskej práce, ak. rok 2020/21, vedenie: Ing. Michal Hatala, PhD.}} 

\author{Karin Kollarovičová\\[2pt]
	{\small Slovenská technická univerzita v Bratislave}\\
	{\small Fakulta informatiky a informačných technológií}\\
	{\small \texttt{xkollarovicovak@stuba.sk}}
	}

\date{\small 22.10.2020}



\begin{document}

\maketitle

\begin{abstract}
	The focus of this paper is on analysing a method called Gamification and its use in the e-learning environment. 
	Gamification plays a big role when it comes to successfully motivating students and improving their social interactions with one another while communicating via the internet. 
	The psychological point of view is crucial to fully understand how educational tools should be developed. 
	According to multiple studies, this method has been practically used in education and has shown some good results. 
	Most of the success exists thanks to well-known mechanics implemented from the video game industry. 
	This article provides arguments on why gamifying education is beneficial as well as a somewhat critical point of view, from which objective idea can be formed. 
\end{abstract}



\section{Introduction} \label{introduction}
Students’ lack of motivation may have always been a problem, but it has never been more noticeable than it is today, in the age of e-learning. 
Solution to this problem as well as many more ideas behind why Gamification is needed and can be helpful will be proposed in section~\ref{section2}. 
Gamifying education has truly proven useful over the past years due to its high interactivity. What makes it so efficient and effective are all the gameplay mechanics if incorporates into learning.
We will take a look at what precisely is understood under the term ‘Gamification’ (section ~\ref{section3}) as well as its practical use (section~\ref{section6} ).
Lack of motivation can easily occur when students and teachers cannot or simply do not interact with one another efficiently. To fully understand how we can benefit from gamifying education we also have to understand the psychology behind it.
Therefore, section~\ref{section4} will be focused on how motivation works and what we can do to boost it. Despite proven useful, there still may be some shortcomings about whether the results achieved by Gamification are good enough to keep up its good reputation. You can read further about these in section ~\ref{section5}.

\section{The lack of motivation} \label{section2}

Z obr.~\ref{f:rozhod} je všetko jasné. 

\begin{figure*}[tbh]
\centering
%\includegraphics[scale=1.0]{diagram.pdf}
Aj text môže byť prezentovaný ako obrázok. Stane sa z neho označný plávajúci objekt. Po vytvorení diagramu zrušte znak \texttt{\%} pred príkazom \verb|\includegraphics| označte tento riadok ako komentár (tiež pomocou znaku \texttt{\%}).
\caption{Rozhodujúci argument.}
\label{f:rozhod}
\end{figure*}



\section{What is gamification} \label{section3}
The term ‘Gamification’ simply means applying specific and efficient gameplay mechanics to the education process. It, however, does not necessarily mean creating full games as we know them.\cite{Raymer}
A bit more precise definition is creating e-learning tools with the use and application of specific elements from games. These elements are what makes videogames fun, while maintaining the challenging and educational character.\cite{Abu-Dawood} 
It must also be remembered that successful gamification is not solely about the game attributes of it. Equally important are the social and cognitive factors, which are the true sources of education. \cite{Raymer}
If executed correctly, the student/learner is given proper motivational, social, and emotional satisfaction.\cite{Abu-Dawood} 
\subsection{Engagement in Gamification} \label{section3:1}



Ten istý zoznam, len číslovaný:

\begin{enumerate}
\item jedna vec
\item druhá vec
	\begin{enumerate}
	\item x
	\item y
	\end{enumerate}
\end{enumerate}


\subsection{Ešte nejaké vysvetlenie} \label{section3:2}
Hello world

\paragraph{Veľmi dôležitá poznámka.}
Niekedy je potrebné nadpisom označiť odsek. Text pokračuje hneď za nadpisom.



\section{What is motivation?} \label{section4}




\section{Gamification in practical use} \label{section5}




\section{Conclusion} \label{section6} % prípadne iný variant názvu



%\acknowledgement{Ak niekomu chcete poďakovať\ldots}

\bibliographystyle{plain}
\bibliography{literatura}

\end{document}
