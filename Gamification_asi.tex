% Metódy inžinierskej práce

\documentclass[10pt,twoside,english,a4paper]{article}

\usepackage[english]{babel}
%\usepackage[T1]{fontenc}
\usepackage[IL2]{fontenc} % lepšia sadzba písmena Ľ než v T1
\usepackage[utf8]{inputenc}
\usepackage{graphicx}
\usepackage{url} % príkaz \url na formátovanie URL
\usepackage{hyperref} % odkazy v texte budú aktívne (pri niektorých triedach dokumentov spôsobuje posun textu)

\usepackage{cite}
%\usepackage{times}

\pagestyle{headings}

\title{Gamification-How to motivate students in an online environment\thanks{Semestrálny projekt v predmete Metódy inžinierskej práce, ak. rok 2020/21, vedenie: Ing. Michal Hatala, PhD.}} 

\author{Karin Kollarovičová\\[2pt]
	{\small Slovenská technická univerzita v Bratislave}\\
	{\small Fakulta informatiky a informačných technológií}\\
	{\small \texttt{xkollarovicovak@stuba.sk}}
	}

\date{\small 22.10.2020}



\begin{document}

\maketitle

\begin{abstract}
	Students’ lack of motivation may have always been a problem, but it has never been more noticeable than it is today, in the age of e-learning. Lack of motivation can easily occur when students and teachers cannot or simply do not interact with one another efficiently. One of the reasons for this may the absence of social interaction while studying via internet. As e-learning becomes more and more popular, many problems arise. In order to effectively eliminate those shortcomings, experts are constantly trying to develop the most effective method for online teaching and learning. One of the solutions to this problem is a method called ‘Gamification’. 	
\end{abstract}



\section{Introduction}
Students’ lack of motivation may have always been a problem, but it has never been more noticeable than it is today, in the age of e-learning. 
Solution to this problem as well as many more ideas behind why Gamification is needed and can be helpful will be proposed in section~\ref{section2}. 
Gamifying education has truly proven useful over the past years due to its high interactivity. What makes it so efficient and effective are all the gameplay mechanics if incorporates into learning.
We will take a look at what precisely is understood under the term ‘Gamification’ (section ~\ref{section3}) as well as its practical use (~\ref{section6} ).
Lack of motivation can easily occur when students and teachers cannot or simply do not interact with one another efficiently. To fully understand how we can benefit from gamifying education we also have to understand the psychology behind it.
Therefore, section~\ref{section4} will be focused on how motivation works and what we can do to boost it. Despite proven useful, there still may be some shortcomings about whether the results achieved by Gamification are good enough to keep up its good reputation. You can read further about these in section ~\ref{section5}.

\section{The lack of motivation} \label{section2}

Z obr.~\ref{f:rozhod} je všetko jasné. 

\begin{figure*}[tbh]
\centering
%\includegraphics[scale=1.0]{diagram.pdf}
Aj text môže byť prezentovaný ako obrázok. Stane sa z neho označný plávajúci objekt. Po vytvorení diagramu zrušte znak \texttt{\%} pred príkazom \verb|\includegraphics| označte tento riadok ako komentár (tiež pomocou znaku \texttt{\%}).
\caption{Rozhodujúci argument.}
\label{f:rozhod}
\end{figure*}



\section{What is gamification} \label{section3}
ahoj\cite{Raymer}
\subsection{Nejaké vysvetlenie} \label{section3:1}

Niekedy treba uviesť zoznam:

\begin{itemize}
\item jedna vec\cite{Abu-Dawood}
\item druhá vec\cite{AL-Smadi}
	\begin{itemize}
	\item x\cite{LiDongUntchChasteen}
	\item y
	\end{itemize}
\end{itemize}

Ten istý zoznam, len číslovaný:

\begin{enumerate}
\item jedna vec
\item druhá vec
	\begin{enumerate}
	\item x
	\item y
	\end{enumerate}
\end{enumerate}


\subsection{Ešte nejaké vysvetlenie} \label{section3:2}
Hello world

\paragraph{Veľmi dôležitá poznámka.}
Niekedy je potrebné nadpisom označiť odsek. Text pokračuje hneď za nadpisom.



\section{What is motivation?} \label{section4}




\section{Gamification in practical use} \label{section5}




\section{Conclusion} \label{section6} % prípadne iný variant názvu



%\acknowledgement{Ak niekomu chcete poďakovať\ldots}

\bibliographystyle{plain}
\bibliography{literatura}

\end{document}
