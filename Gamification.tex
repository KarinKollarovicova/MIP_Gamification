\documentclass[10pt,english,a4paper]{article}
\usepackage[english]{babel}
\usepackage[IL2]{fontenc} 
\usepackage[utf8]{inputenc}
\usepackage{graphicx}
\usepackage{url} 
\usepackage{hyperref}
\usepackage{cite}

\pagestyle{headings}

\title{Gamification-How to motivate students in an online environment\thanks{Semestrálny projekt v predmete Metódy inžinierskej práce, ak. rok 2020/21, vedenie: Ing. Michal Hatala, PhD.}} 

\author{Karin Kollarovičová\\[2pt]
	{\small Slovak University of Technology in Bratislava}\\
	{\small Faculty of Informatics and Information Technologies}\\
	{\small \texttt{xkollarovicovak@stuba.sk}}
	}

\date{\small 28.11.2020}



\begin{document}

\maketitle

\begin{abstract}
	The focus of this paper is on analysing a method called Gamification and its use in the e-learning environment. 
	Gamification plays a big role when it comes to successfully motivating students and improving their social interactions with one another while learning via internet. 
	The psychological point of view is crucial to fully understand how educational tools should be developed. 
	According to multiple studies, this method has been practically used in education and has shown some good results. 
	Most of the success exists thanks to well-known mechanics implemented from the video game industry. 
	This article provides arguments on why gamifying education is beneficial and how it works. 
\end{abstract}

\section{Introduction} \label{introduction}
	Students’ lack of motivation may have always been a problem, but it has never been more noticeable than it is today, in the age of e-learning. 
	Online education is often poorly organized and does not provide the same satisfaction as in-person learning does.
	Lack of motivation can easily occur when students and teachers cannot or simply do not interact with one another efficiently. To fully understand how we can benefit from gamifying education we also have to understand the psychology behind it. 
	Therefore, section ~\ref{section2} will be focused on how motivation works and what can be done to boost it.
	Gamifying education has truly proven useful over the past years due to its high interactivity. What makes it so efficient and effective are all the gameplay mechanics if incorporates into learning.
	We will take a look at what precisely is understood under the term ‘Gamification’ (section ~\ref{section3}) as well as its practical use (section~\ref{section4}).

\section{The lack of motivation} \label{section2}
	To understand how important motivation is and what can be done to improve it (using Gamification), we have to take a look at it from a slightly psychological point of view. First, we have to understand what motivation actually means. 
	To motivate is to create an \textit{“energizing force that initiates and sustains behaviour and ultimately produces results,”} writes Guyan \cite{Guyan}. 

	Basically, there are 2 main types of motivation: extrinsic and intrinsic(view Figure \ref{fig:motivation}). Extrinsic motivation can be found in 4 different forms. 
	According to Edward Deci and Richard Ryan's Self-determination theory (SDT) and its sub-theory called Organismic integration theory (OIT), the decision-making factor which decides what form each one represents is the level of external control it has over an individual. 
	It varies: from being fully external (e.g. getting paid, not receiving punishment) to being somewhat external where we can talk about people doing something simply to avoid feeling guilt or shame. Other forms of extrinsic motivation are more internally oriented. 
	In this case, the individual either sees the importance and wants to get it done correctly or feels connection between the task and his/her beliefs. Here we can talk about an internal source of motivation.
	
	Intrinsic motivation, on the other hand, comes from the inside. It depends on the enjoyment of the task performed. Simply said, people feel intrinsic (fully internal) motivation once they do something voluntarily and genuinely enjoy it.

	\begin{figure}[htp]
		\centering
		\includegraphics[scale=0.23]{Motivacia.png}
		\caption{The types of motivation according to the Self determination theory}
		\label{fig:motivation}
	\end{figure}
	

	Now to transform this into the online learning problematics. Here is what can help students find not only their intrinsic motivation (as it is the most important and long-lasting one), but also help them stay engaged. 
	Competence, autonomy and relatedness. If these three are provided, students will be on the right path to finding enjoyment in the online education process\cite{Guyan}. Precise strategies will be provided in section ~\ref{section3.1}.

\section{What is gamification} \label{section3}
	The term ‘Gamification’ simply means applying specific and efficient gameplay mechanics to the education process. It, however, does not necessarily mean creating full games as we know them\cite{Raymer}.
	A bit more precise definition is creating e-learning tools with the use and application of specific elements from games. These elements are what makes videogames fun, while maintaining the challenging character\cite{Abu-Dawood}. 
	It must also be remembered that successful gamification is not solely about the game attributes of it. Equally important are the social and cognitive factors, which are the true sources of education \cite{Raymer}.
	If executed correctly, the student/learner is given proper motivational, social, and emotional satisfaction\cite{Abu-Dawood}.

\subsection{Engagement and game mechanics in Gamification} \label{section3.1}
	Having user’s full attention or making sure they stay engaged for a longer period of time is something game developers have been mastering for decades. 
	Their knowledge can therefore be helpful. 
	According to Raymer \cite{Raymer}, \textit{“essentially, there are two components to the perception of something being rewarding: wanting and liking”.} 
	To put it in other words, being rewarded for something we enjoy and like is what keeps us engaged, excited and motivated. 
	This must naturally be remembered in the development process of a certain gamification tool.

	To ensure the student stays active and wide awake, we must firstly think of interactivity. 
	If a certain task is expected to be performed, it should not only be practical, but also fun. 
	The interactivity, in this case, could mean sufficient communication between a learner and an e-learning tool. 
	Creativity plays a part in this too. Pressing the same grey button for an hour in order to progress simply does not count as interactive or engaging learning \cite{ AL-Smadi}. 
	
	Another method is to divide study materials into blocks and set some short- and long-term goals. 
	This is fully implemented from videogames, where players go through challenges, fights and slowly build and perfect their character by successfully completing them.
	For those, immediate reward can be received, which provides great feedback to the player/ student.
	Once finished with the preparation, students are confident and ready to encounter the final boss, which we can refer to as an important (or final) exam. 
	Being well prepared eliminates stress and discomfort before an exam and makes it easier for students to concentrate.
	Therefore, performing tasks with enlarging difficulty is essential. Once the tasks are too easy, students become bored and loose interest. 

	In contrast to that, having too complex problems to solve (especially in the beginning) is contra productive and may lead to frustration. 
	Clear instructions are particularly important to avoid confusion and ensure high efficiency \cite{Raymer}. 
	Frustration can also be eliminated by the tasks being relatable and customizable. 
	This can trigger positive emotions and give the learner opportunities to experience something unique \cite{ AL-Smadi}.
	
	The graduality of the tasks can often become predictable though.
	This is why it is always a good idea to let the learner choose his/her own path using a so called “Nonlinear Goal Progression” or include some surprise elements (bonus task with higher difficulty).
	By having the opportunity to customize the whole learning process, it becomes more friendly for the students to use. 
	This is very efficient for the student. On contrary, for the developer, it is extra work, therefore the cost of such software can climb quite high. \cite{Raymer}.

\section{Gamification in use} \label{section4}
	Gamification has been proven particularly useful over the past few years.
	As an example, we can take a look at some of the most famous apps or websites in which gamification techniques are used.

		 Duolingo (platform for learning languages) includes scoreboards, daily challenges or experience points. 
		 It keeps you on track with your achievements and provides direct feedback. 
		 
		 Kahoot (online platform for quizzes) is a very interactive tool to help teachers in classes. 
		 It has a nice environment and can be modified to fit the teacher's needs.
		 
		 Scratch is a creative coding tool which can be accessed through a website. 
		 The programming itself is done by assembling blocks of pre-made code, allowing even young students to create their own games or animations. 
		 It is very intuitive and includes an animated representation of your already made code by using real time animation, which is a form of immediate feedback.

	To prove the theory, this following experiment provides useful information. 
	A team of reseachers used an online networking environment called PeerSpace, which was designed to help with social interaction and engagement among college students.
	It includes chat rooms, profiles, forum or online repositories for students to use. 
	\textit{"The goal is to improve student motivation in learning through the building of supportive peer networks"}, states the study.
	
	Using two control groups, the researchers allowed the first group of students to use the gamification features included in PeerSpace, while the second group was denied this option.
	Some of the gamification features are: simple games ("Who am I?- played using student's and tacher's photos), experience points, a leveling system. 
	This created higher activity in posting questions or participating in discussions. Overall the first group became three times more active than the second group\cite{Li}. 
	
	We can clearly see how it affected students from the first group in a poisitve way. 
	They were encouraged to get to know each other, communicate and therefore, were also able to help each other and make the whole education process more fun.

\section{Conclusion} \label{section5}
	To sum it all up, we can say that Gamification has a huge potential when it comes to keeping students engaged. 
	It helps them with the realisation that e-learning can be enjoyable.
	It also solves many other struggles such as exams taking too long to be corrected or highly inefficient communication or big skill differences among classmates.
	
	Overall, Gamification has some great features and presents a modern approach to studying. 
	Thanks to all the similarities it shares with activities most teenagers and kids enjoy doing in their free time, it helps them to transfer this positive connection from gaming to e-learning. 
	What should be remembered though is, that the human factor is always inevitable. What works for one person does not automatically have to work for another one.
	Although with all the possibilities Gamification creates, it is very likely that each person finds what works for them, therefore, it can be considered a very efficient method when it comes to motivating and engaging students in an online environment.

\bibliographystyle{ieeetr}
\bibliography{literatura}

\end{document}
