% Metódy inžinierskej práce

\documentclass[10pt,twoside,english,a4paper]{article}

\usepackage[english]{babel}
\usepackage[IL2]{fontenc} 
\usepackage[utf8]{inputenc}
\usepackage{graphicx}
\usepackage{url} 
\usepackage{hyperref}
\usepackage{cite}

\pagestyle{headings}

\title{Gamification-How to motivate students in an online environment\thanks{Semestrálny projekt v predmete Metódy inžinierskej práce, ak. rok 2020/21, vedenie: Ing. Michal Hatala, PhD.}} 

\author{Karin Kollarovičová\\[2pt]
	{\small Slovenská technická univerzita v Bratislave}\\
	{\small Fakulta informatiky a informačných technológií}\\
	{\small \texttt{xkollarovicovak@stuba.sk}}
	}

\date{\small 29.10.2020}



\begin{document}

\maketitle

\begin{abstract}
	The focus of this paper is on analysing a method called Gamification and its use in the e-learning environment. 
	Gamification plays a big role when it comes to successfully motivating students and improving their social interactions with one another while learning via internet. 
	The psychological point of view is crucial to fully understand how educational tools should be developed. 
	According to multiple studies, this method has been practically used in education and has shown some good results. 
	Most of the success exists thanks to well-known mechanics implemented from the video game industry. 
	This article provides arguments on why gamifying education is beneficial as well as a somewhat critical point of view, from which objective idea can be formed. 
\end{abstract}



\section{Introduction} \label{introduction}
	Students’ lack of motivation may have always been a problem, but it has never been more noticeable than it is today, in the age of e-learning. 
	Especially during the COVID-19 pandemic, many more student and teachers are the victims of a non-functional online education system which is poorly organized and does not provide the same satisfaction as in-person learning does.
	Lack of motivation can easily occur when students and teachers cannot or simply do not interact with one another efficiently. To fully understand how we can benefit from gamifying education we also have to understand the psychology behind it. 
	Therefore, section ~\ref{section2} will be focused on how motivation works and what can be done to boost it.
	Gamifying education has truly proven useful over the past years due to its high interactivity. What makes it so efficient and effective are all the gameplay mechanics if incorporates into learning.
	We will take a look at what precisely is understood under the term ‘Gamification’ (section ~\ref{section3}) as well as its practical use (section~\ref{section4}).
	Despite proven useful, there still may be some shortcomings about whether the results achieved by Gamification are good enough to keep up its good reputation. You can read further about these in section ~\ref{section5}.

\section{The lack of motivation} \label{section2}
	To understand how important motivation is and what can be done to improve it (using Gamification), we have to take a look at it from a slightly psychological point of view. First, we have to understand what motivation actually means. 
	To motivate is to create an “energizing force that initiates and sustains behaviour and ultimately produces results,” writes Guyan \cite{Guyan}. 

	Basically, there are 2 main types of motivation: extrinsic and intrinsic. Exttrinsic motivation can be seen in 4 different forms. 
	According to Edward Deci and Richard Ryan's self-determination theory (SDT) and its sub-theory called organismic integration theory (OIT), the decision-making factor which decides what form each one represents is the level of external control it has over an individual. 
	It varies: from being fully external (e.g. getting paid, not receiving punishment) to being somewhat external where we can talk about people avoiding feeling guilt or shame. Other forms of extrinsic motivation are more internally oriented. 
	In this case, the individual either sees the importance and wants to get it done correctly or feels connection to his/her beliefs. Here we can talk about an internal source of motivation.
	
	Intrinsic motivation, on the other hand, comes from the inside. It depends on the enjoyment of the task performed. Simply said, people feel intrinsic (fully internal) motivation once they do it voluntarily and genuinely enjoy it.

	Now to transform this into our problematic with online learning. Here is what can help learners find not only their intrinsic motivation (as it is the most important and long-lasting one), but also to help them stay engaged. 
	Competence, autonomy and relatedness. If these three are provided, students will be on the right path to finding enjoyment in the online education process. Precise strategies will be provided in section ~\ref{section3.1}. \cite{Guyan}




\section{What is gamification} \label{section3}
	The term ‘Gamification’ simply means applying specific and efficient gameplay mechanics to the education process. It, however, does not necessarily mean creating full games as we know them.\cite{Raymer}
	A bit more precise definition is creating e-learning tools with the use and application of specific elements from games. These elements are what makes videogames fun, while maintaining the challenging and educational character.\cite{Abu-Dawood} 
	It must also be remembered that successful gamification is not solely about the game attributes of it. Equally important are the social and cognitive factors, which are the true sources of education. \cite{Raymer}
	If executed correctly, the student/learner is given proper motivational, social, and emotional satisfaction.\cite{Abu-Dawood} 

\subsection{Engagement and game mechanincs in Gamification} \label{section3.1}
	Having user’s full attention or making sure they stay engaged for a longer period of time is something game developers have been mastering for decades. Their knowledge can therefore be helpful. 
	According to Raymer \cite{Raymer}, “essentially, there are two components to the perception of something being rewarding: wanting and liking”. 
	To put it in other words, being rewarded for something we enjoy and like is what keeps us engaged, excited and motivated. 
	This must naturally be remembered in the development process of a certain gamification tool.

	To ensure the student stays active and wide awake, we must think of interactivity. If a certain task is expected to be performed, it should not only be practical, but also fun. 
	The interactivity means sufficient communication between learner and the e-learning tool. Creativity plays a part in this too. Pressing the same button for an hour in order to progress simply does not count as interactive or engaging learning. \cite{ AL-Smadi} 
	
	Second method is to divide study materials into short- and long-term goals. The more practice students get the better. This is fully implemented from videogames, where players slowly build and perfect their character to be able to encounter the final boss. 
	Performing tasks with enlarging difficulty is essential. Once the tasks are too easy, students become bored and loose interest. In contrast to that, having too complex problems to solve (especially at the beginning) is contra productive as well. 
	As a result of that, frustration leading to loss of motivation may appear. Efficiency becomes higher when their gamification experience does not include the feeling of frustation caused by obscure instructions. Clear instructions are particularly important. \cite{Raymer} 
	Frustration can be eliminated also by the tasks staying relatable and customizable. This can trigger positive emotions and give the learner opportunities to experience something unique, something worth exploring. \cite{ AL-Smadi}
	
	The graduality of the tasks can often become predictable though. And yet again, boredom and lack of surprise can occur. This is why it is always a good idea to let the learner choose his/her own path, creating a so called “Nonlinear Goal Progression”. Not everyone learns everything at the same pace.
	This is very efficient for the student. On contrary, for the developer it is extra work, therefore the cost of such software can climb quite high. \cite{Raymer}.


\subsection{Game mechanics} \label{section3.2}



\section{Gamification in use} \label{section4}



\section{Critical view} \label{section5}



\section{Conclusion} \label{section6}



\bibliographystyle{plain}
\bibliography{literatura}

\end{document}
